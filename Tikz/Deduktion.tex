\documentclass[border=10pt]{standalone}
\usepackage{smartdiagram}
\usepackage[ngerman]{babel}
\usepackage[utf8]{inputenc} %UTF-Code für Umlaute
\usepackage[T1]{fontenc} %Trennung von Wörtern mit Umlauten
\usepackage{lmodern}
%\usepackage{libertine}
\renewcommand*\familydefault{\sfdefault} %Serifenlose Schrift als Standard
\usepackage{microtype}
\usesmartdiagramlibrary{additions}

\usetikzlibrary{fit}

\tikzstyle{container} = [draw, rectangle, semithick, %inner sep=0cm
]

%Aufzählungsstriche
\AtBeginDocument{
	\def\labelitemi{\normalfont\bfseries{--}}
}


\begin{document}
	
	\begin{tikzpicture}[
	every node/.style = {shape=rectangle, % is not necessary, default node's shape is rectangle
		rounded corners,
		draw= none	, semithick,
	%	text width=5cm,
		align=center,
		node distance=1.6cm
	}
	]
	\node (gegeben-D1)[]{{gegeben}};
	\node[right = of gegeben-D1](gesucht-I1){\textcolor{red}{gesucht}};
	\node[right = 0.3cm of gesucht-I1, draw = none](Mittelpunkt){};
	\node[right = 0.3cm of Mittelpunkt](gegeben-I1){{gegeben}};
	\node[right = of gegeben-I1](gesucht-A1){\textcolor{red}{gesucht}};
	
%Überschriften
	
	\node[above = of gegeben-D1](Deduktion){\textbf{logische Deduktion}};
	\node[above = 1.75cm of Mittelpunkt](Induktion){\textbf{unvollständige Induktion}};
	\node[above = of gesucht-A1](Abduktion){\textbf{Abduktion}};
	
	\node[below = of gegeben-D1](gegeben-D2){{gegeben}};
	\node[below = of gesucht-I1](gegeben-I2){{gegeben}};
	\node[below = of gegeben-I1](gesucht-I2){\textcolor{red}{gesucht}};
	\node[below = of gesucht-A1](gesucht-A2){\textcolor{red}{gesucht}};
	
	\node[below = of gegeben-D2](gesucht-D3){\textcolor{red}{gesucht}};
	\node[below = of gegeben-I2](gegeben-I3a){{gegeben}};
	\node[below = of gesucht-I2](gegeben-I3b){{gegeben}};
	\node[below = of gesucht-A2](gegeben-A3){{gegeben}};
	
%Linke Spalte
	
	\node[left = of gegeben-D1](Antezedenz){Antezedenzbedingungen};
	\node[left = of gegeben-D2](Gesetze){gesetzesartige Aussagen};
	\node[below = of Gesetze, draw, text width=4cm](Explanandum){\textbf{Explanandum}};
	
	\node (Explanans) [container,fit=(Gesetze)(Antezedenz), text width=4cm]{\textbf{Explanans}};
	\node () [container,fit=(gegeben-D1)(gegeben-D2)(gesucht-A1)]{};
	\node () [container,fit=(gesucht-D3)(gegeben-A3)]{};
%	\node () [container,fit=(gegeben-D1)(gesucht-D3)]{};

%Pfade
	
	\path [->, thick] (gegeben-D1) edge (gegeben-D2);
	\path [->, thick] (gegeben-D2) edge (gesucht-D3);	
	
	\path [->, thick] (gegeben-I3a) edge (gegeben-I2);
	\path [->, thick] (gegeben-I2) edge (gesucht-I1);	
	
	\path [->, thick] (gegeben-I3b) edge (gesucht-I2);
	\path [->, thick] (gegeben-I1) edge (gesucht-I2);
		
	\path [->, thick] (gegeben-A3) edge (gesucht-A2);
	\path [->, thick] (gesucht-A2) edge (gesucht-A1);
	

	
%Überschrift
	\path [draw=none] (Antezedenz) -- (gesucht-A1) node [midway](Mittelpunkt2){};
	
	\node[above= 3cm of Mittelpunkt2]{\huge Suchrichtungen von Erklärungen};	
	
\end{tikzpicture}
	
\end{document}